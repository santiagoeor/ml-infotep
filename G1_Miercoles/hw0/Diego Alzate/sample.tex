\documentclass[a4paper, 10pt]{article}
\hyphenpenalty=8000
\textwidth=125mm
\textheight=185mm

\usepackage{graphicx}
%this package is flexible for image insertion
%
\usepackage{alltt}
%this package is suitable for the description of algorithms and computer programs
%
\usepackage{amsmath}
%this package draws mathematical symbols smoothly
%
\usepackage[hidelinks, pdftex]{hyperref}
%this package produces hypertext links in the document

\pagenumbering{arabic}
\setcounter{page}{1}
\renewcommand{\thefootnote}{\fnsymbol{footnote}}
\newcommand{\doi}[1]{\href{https://doi.org/#1}{\texttt{https://doi.org/#1}}}

\begin{document}

\begin{center}
Nonlinear Analysis: Modelling and Control, Vol. 01, No. 01, 2025\02\28
\copyright\ Infotep\\[24pt]
\LARGE
\textbf{Hallando el area de una mancha}\
\small
\textbf {Diego Fernando Alzate Alvarez}\\[6pt]
Institution(s) of author(s), address(es) \\ author@somewhere.host\\[6pt]
Received: date\quad/\quad
Revised: date\quad/\quad
Publised online: data
\end{center}

\begin{abstract}
En este trabajo, se tiene como objetivo obtener el área de una mancha en una imagen suministrada por el profesor. Para ello, se utiliza un enfoque basado en la generación aleatoria de puntos dentro de la imagen y el cálculo de la proporción de puntos que caen dentro de la mancha. El proceso se describe a continuación:
 \vskip 2mm

\end{abstract}

\nocite{2009ProcDETAp}

\section{Procedimiento}
Se asignan de manera aleatoria $n$ puntos dentro de la imagen, cuyos valores son generados aleatoriamente dentro de la imagen. Luego, se obtiene la cantidad de puntos que cayeron dentro de la mancha y el tamaño de la imagen, su ancho y alto. 
\section{Fórmula para el Cálculo del Área}
Para obtener el área de la mancha, se utiliza una fórmula compartida por el profesor, que es la siguiente:

\[
\text{Área} = \lim_{n \to \infty} \left( \frac{\text{p}}{n} \right) \times \text{wh}
\]

Donde:
\begin{itemize}
    \item $n$: es el número total de puntos generados aleatoriamente.
    \item \text{p}: es el número de puntos que caen dentro de la mancha, según el umbral.
\item \text{w}: Ancho de la imagen.
\item \text{h}: Alto de la imagen.
    \end{itemize}

\section*{Resultado del programa}

Para evaluar el comportamiento del programa con diferentes tamaños de muestra, se realizaron tres pruebas con valores de \( n \), siendo estos 10, 100 y 1000. Los resultados obtenidos fueron los siguientes:

\begin{itemize}
    \item \textbf{Cuando \( n = 10 \):}
    \begin{itemize}
        \item La cantidad de puntos que cayeron dentro de la mancha fue constante en 7 para todas las iteraciones.
        \item El área promedio calculada fue de 175,000.
    \end{itemize}
    
    \item \textbf{Cuando \( n = 100 \):}
    \begin{itemize}
        \item En este caso, los resultados mostraron un mayor cambio. La cantidad de puntos dentro de la mancha varió entre 83 y 85.
        \item El área promedio obtenida fue de 207,500.
    \end{itemize}
    
    \item \textbf{Cuando \( n = 1000 \):}
    \begin{itemize}
        \item Para \( n = 1000 \), la cantidad de puntos dentro de la mancha variaba un poco más, oscilando entre 779 y 800.
        \item El área promedio calculada fue de 196,750.
    \end{itemize}
\end{itemize}

\section*{Conclusión}

Podemos concluir que, al aumentar el valor de \( n \), el área estimada oscila entre los valores obtenidos para \( n = 10 \) y \( n = 100 \). Esto sugiere que el valor más preciso del área podría estar dentro de este rango. Además, para obtener resultados más precisos, se recomienda aumentar el valor de \( n \), ya que una mayor cantidad de puntos tiende a mejorar la aproximación del área.

\end{document}
