\documentclass[a4paper, 10pt]{article}
\hyphenpenalty=8000
\textwidth=125mm
\textheight=185mm

\usepackage{graphicx} % Para insertar imágenes
\usepackage{alltt}    % Para algoritmos y código
\usepackage{amsmath}  % Para símbolos matemáticos
\usepackage[hidelinks, pdftex]{hyperref}

\pagenumbering{arabic}
\setcounter{page}{1}
\renewcommand{\thefootnote}{\fnsymbol{footnote}}

\begin{document}

\begin{center}
\LARGE \textbf{Análisis de Convergencia de K-Means en Datos Sintéticos}\\[12pt]
\small \textbf{Alan Miranda. Marlon Caviedes, Cristian Orozco} \\
Infotep \\
\end{center}

% ----------------------------------------------------------------
% Experimentos 1-4 (Baja y Media Dimensión)
% ----------------------------------------------------------------
\section*{Convergencia en Baja y Media Dimensión (Experimentos 1--4)}
\textbf{Objetivo y Metodología:}\\[4pt]
Se aplicó el algoritmo K-Means a 1000 muestras sintéticas para evaluar el número de iteraciones hasta la convergencia en distintas configuraciones de dimensión y número de clusters. Se realizaron 5 repeticiones para cada experimento, promediando las iteraciones obtenidas.\\[4pt]

\textbf{Configuraciones:}
\begin{itemize}
  \item \textbf{Experimento 1 (2D, $k=3$, Euclidiana):}\\
        Ejemplo de iteraciones: R1=4, R2=4, R3=4, R4=8, R5=3 (Promedio $\approx$ 4.6).
  \item \textbf{Experimento 2 (3D, $k=3$, Euclidiana):}\\
        Ejemplo de iteraciones: R1=4, R2=14, R3=4, R4=5, R5=4 (Promedio $\approx$ 6.2).
  \item \textbf{Experimento 3 (10D, $k=3$, Euclidiana):}\\
        Se generó una gráfica de barras con los promedios de iteraciones obtenidos para distintos valores de \texttt{max\_iter}: 2.80, 6.80, 4.40, 11.60.
  \item \textbf{Experimento 4 (100D, $k=3$, Euclidiana):}\\
        Resultados: \texttt{max\_iter}=10: 3.00, 100: 3.20, 1000: 4.40, 10000: 3.20 iteraciones promedio.
\end{itemize}


\textbf{Conclusiones}
\begin{itemize}
  \item En 2D y 3D la convergencia es rápida (entre 3 y 7 iteraciones, en promedio).
  \item En 10D y 100D se observa mayor variabilidad en el número de iteraciones, lo que se atribuye a la complejidad del espacio y la aleatoriedad en la inicialización.
  \item Aumentar el valor de \texttt{max\_iter} no necesariamente reduce el número de iteraciones, ya que la convergencia ocurre antes y depende de la posición inicial de los centroides.
\end{itemize}

\newpage

% ----------------------------------------------------------------
% Experimentos 5 y 6 (Variación de k y Distancias)
% ----------------------------------------------------------------
\section*{Influencia de $k$ y de la Métrica de Distancia (Experimentos 5 y 6)}
\textbf{Experimento 5: Variación de $k$ en 2D}\\[2pt]
Se evaluó el efecto de variar el número de clusters en datos 2D con 1000 muestras, usando $k=2$, $k=3$, $k=5$ y $k=10$. Cada configuración se repitió 5 veces.
\begin{itemize}
  \item \textbf{$k=2$}: Iteraciones: R1=3, R2=6, R3=6, R4=4, R5=3 (Promedio $\approx$ 4.4).
  \item \textbf{$k=3$}: Iteraciones: R1=13, R2=5, R3=4, R4=5, R5=7 (Promedio $\approx$ 6.8).
  \item \textbf{$k=5$}: Iteraciones: R1=39, R2=9, R3=18, R4=14, R5=17 (Promedio $\approx$ 19.4).
  \item \textbf{$k=10$}: Iteraciones: R1=4, R2=24, R3=19, R4=19, R5=21 (Promedio $\approx$ 17.4).
\end{itemize}

\vspace{5mm}
\textbf{Experimento 6: Comparación de Métricas de Distancia en 2D ($k=5$)}\\[2pt]
Se compararon tres métricas de distancia en 2D para $k=5$ (5 repeticiones):
\begin{itemize}
  \item \textbf{Euclidiana}: Iteraciones: R1=9, R2=14, R3=36, R4=27, R5=11 (Promedio $\approx$ 19.4).
  \item \textbf{Manhattan}: Iteraciones: R1=18, R2=6, R3=10, R4=13, R5=9 (Promedio $\approx$ 11.2).
  \item \textbf{Mahalanobis}: Iteraciones: R1=17, R2=8, R3=22, R4=10, R5=18 (Promedio $\approx$ 15).
\end{itemize}

\textbf{Conclusiones}
\begin{itemize}
  \item La variación de $k$ influye en la complejidad: al aumentar $k$, en general se requiere un mayor número de iteraciones, aunque existe variabilidad entre repeticiones.
  \item En el Experimento 6, la métrica Manhattan mostró convergencia más rápida (menor promedio de iteraciones) que Euclidiana y Mahalanobis.
  \item La elección de la métrica y el número de clusters tiene un impacto significativo en la convergencia y en la formación final de clusters.
\end{itemize}


\end{document}
