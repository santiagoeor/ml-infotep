\documentclass[a4paper, 10pt]{article}
\usepackage{graphicx}
\usepackage{amsmath}
\usepackage[hidelinks]{hyperref}
\usepackage{enumitem}

\begin{document}

\title{Pruebas con Datos Sintéticos en Experimentos de Clustering}
\author{alexdith ariza
nelson martinez}
\date{\today}
\maketitle

\section{Introducción}
Este documento describe los experimentos realizados con datos sintéticos para evaluar el rendimiento del algoritmo de clustering K-Means bajo diferentes configuraciones de parámetros.

\section{Experimentos}
Se realizaron los siguientes experimentos:

\begin{enumerate}
    \item \textbf{(0.5 punto)} 1000 muestras, 2D, $k=3$, distancia euclidiana, máximas iteraciones = 10, 100, 1000, 10000.
    \item \textbf{(0.5 punto)} 1000 muestras, 3D, $k=3$, distancia euclidiana, máximas iteraciones = 10, 100, 1000, 10000.
    \item \textbf{(0.5 punto)} 1000 muestras, 10D, $k=3$, distancia euclidiana, máximas iteraciones = 10, 100, 1000, 10000.
    \item \textbf{(0.5 punto)} 1000 muestras, 100D, $k=3$, distancia euclidiana, máximas iteraciones = 10, 100, 1000, 10000.
    \item \textbf{(0.5 punto)} 1000 muestras, 3D, $k=2,3,5,10$, distancia euclidiana, máximas iteraciones = 1000.
    \item \textbf{(1 punto)} 1000 muestras, 2D, $k=5$, distancia euclidiana, Manhattan y Mahalanobis, máximas iteraciones = 1000.
    \item \textbf{(2 puntos)} Determinar para cada experimento cuántas iteraciones son necesarias para la convergencia (cuando los centroides actuales sean iguales a los centroides anteriores).
    \item \textbf{(1 punto)} Presentación de resultados.
\end{enumerate}

\section{Resultados y Análisis}
En esta sección se incluirán los resultados obtenidos en cada experimento y un análisis sobre el impacto de la dimensionalidad, el número de clusters y las métricas de distancia en la convergencia del algoritmo.

\end{document}
